
% Редактируем: конфигурация, личные настройки: имя, название предмета и пр. для титульной страницы и метаданных документа здесь



%%% Работа с русским языком
\usepackage[english,russian]{babel}   %% загружает пакет многоязыковой вёрстки
\usepackage{fontspec}      %% подготавливает загрузку шрифтов Open Type, True Type и др.
\defaultfontfeatures{Ligatures={TeX},Renderer=Basic}  %% свойства шрифтов по умолчанию
\setmainfont[Ligatures={TeX,Historic}]{Times New Roman} %% задаёт основной шрифт документа
\setsansfont{Comic Sans MS}                    %% задаёт шрифт без засечек
\setmonofont{Courier New}

%\include{include/settings.tex}



% рабочие ссылки в документе
\usepackage{hyperref}
\urlstyle{same}
% графика
\usepackage{graphicx}
\usepackage{tikz}
%\usepackage{pgfplots}

% качественные листинги кода
%\usepackage{minted}
%\usepackage{listings}
%\usepackage{lstfiracode}


% библиография
\bibliographystyle{templates/gost-numeric.bbx}
\usepackage{csquotes}
\usepackage[parentracker=true,backend=biber,hyperref=true,bibencoding=utf8,style=numeric-comp,language=auto,autolang=other,citestyle=gost-numeric,defernumbers=true,bibstyle=gost-numeric,sorting=ntvy]{biblatex}

% для заголовков
\usepackage{caption} 

% разное для математики
\usepackage{amsmath, amsfonts, amssymb, amsthm, mathtools}

% водяной знак на документе, см. main.tex
%\usepackage[printwatermark]{xwatermark} 

% для презентаций
\usepackage{here}
\usepackage{animate}
\usepackage{bm}

