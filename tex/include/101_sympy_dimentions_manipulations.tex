\documentclass[10pt]{article}

    \usepackage[breakable]{tcolorbox}
    \usepackage{parskip} % Stop auto-indenting (to mimic markdown behaviour)
    

\usepackage[english,russian]{babel}   %% загружает пакет многоязыковой вёрстки
\usepackage{fontspec}      %% подготавливает загрузку шрифтов Open Type, True Type и др.
\defaultfontfeatures{Ligatures={TeX},Renderer=Basic}  %% свойства шрифтов по умолчанию
\setmainfont[Ligatures={TeX,Historic}]{Times New Roman} %% задаёт основной шрифт документа
\setsansfont{Comic Sans MS}                    %% задаёт шрифт без засечек
\setmonofont{Courier New}

    % Basic figure setup, for now with no caption control since it's done
    % automatically by Pandoc (which extracts ![](path) syntax from Markdown).
    \usepackage{graphicx}
    % Keep aspect ratio if custom image width or height is specified
    \setkeys{Gin}{keepaspectratio}
    % Maintain compatibility with old templates. Remove in nbconvert 6.0
    \let\Oldincludegraphics\includegraphics
    % Ensure that by default, figures have no caption (until we provide a
    % proper Figure object with a Caption API and a way to capture that
    % in the conversion process - todo).
    \usepackage{caption}
    \DeclareCaptionFormat{nocaption}{}
    \captionsetup{format=nocaption,aboveskip=0pt,belowskip=0pt}

    \usepackage{float}
    \floatplacement{figure}{H} % forces figures to be placed at the correct location
    \usepackage{xcolor} % Allow colors to be defined
    \usepackage{enumerate} % Needed for markdown enumerations to work
    \usepackage{geometry} % Used to adjust the document margins
    \usepackage{amsmath} % Equations
    \usepackage{amssymb} % Equations
    \usepackage{textcomp} % defines textquotesingle
    % Hack from http://tex.stackexchange.com/a/47451/13684:
    \AtBeginDocument{%
        \def\PYZsq{\textquotesingle}% Upright quotes in Pygmentized code
    }
    \usepackage{upquote} % Upright quotes for verbatim code
    \usepackage{eurosym} % defines \euro

    \usepackage{iftex}
    \ifPDFTeX
        \usepackage[T1]{fontenc}
        \IfFileExists{alphabeta.sty}{
              \usepackage{alphabeta}
          }{
              \usepackage[mathletters]{ucs}
              \usepackage[utf8x]{inputenc}
          }
    \else
        \usepackage{fontspec}
        \usepackage{unicode-math}
    \fi

    \usepackage{fancyvrb} % verbatim replacement that allows latex
    \usepackage{grffile} % extends the file name processing of package graphics
                         % to support a larger range
    \makeatletter % fix for old versions of grffile with XeLaTeX
    \@ifpackagelater{grffile}{2019/11/01}
    {
      % Do nothing on new versions
    }
    {
      \def\Gread@@xetex#1{%
        \IfFileExists{"\Gin@base".bb}%
        {\Gread@eps{\Gin@base.bb}}%
        {\Gread@@xetex@aux#1}%
      }
    }
    \makeatother
    \usepackage[Export]{adjustbox} % Used to constrain images to a maximum size
    \adjustboxset{max size={0.9\linewidth}{0.9\paperheight}}

    % The hyperref package gives us a pdf with properly built
    % internal navigation ('pdf bookmarks' for the table of contents,
    % internal cross-reference links, web links for URLs, etc.)
    \usepackage{hyperref}
    % The default LaTeX title has an obnoxious amount of whitespace. By default,
    % titling removes some of it. It also provides customization options.
    \usepackage{titling}
    \usepackage{longtable} % longtable support required by pandoc >1.10
    \usepackage{booktabs}  % table support for pandoc > 1.12.2
    \usepackage{array}     % table support for pandoc >= 2.11.3
    \usepackage{calc}      % table minipage width calculation for pandoc >= 2.11.1
    \usepackage[inline]{enumitem} % IRkernel/repr support (it uses the enumerate* environment)
    \usepackage[normalem]{ulem} % ulem is needed to support strikethroughs (\sout)
                                % normalem makes italics be italics, not underlines
    \usepackage{soul}      % strikethrough (\st) support for pandoc >= 3.0.0
    \usepackage{mathrsfs}
    

    
    % Colors for the hyperref package
    \definecolor{urlcolor}{rgb}{0,.145,.698}
    \definecolor{linkcolor}{rgb}{.71,0.21,0.01}
    \definecolor{citecolor}{rgb}{.12,.54,.11}

    % ANSI colors
    \definecolor{ansi-black}{HTML}{3E424D}
    \definecolor{ansi-black-intense}{HTML}{282C36}
    \definecolor{ansi-red}{HTML}{E75C58}
    \definecolor{ansi-red-intense}{HTML}{B22B31}
    \definecolor{ansi-green}{HTML}{00A250}
    \definecolor{ansi-green-intense}{HTML}{007427}
    \definecolor{ansi-yellow}{HTML}{DDB62B}
    \definecolor{ansi-yellow-intense}{HTML}{B27D12}
    \definecolor{ansi-blue}{HTML}{208FFB}
    \definecolor{ansi-blue-intense}{HTML}{0065CA}
    \definecolor{ansi-magenta}{HTML}{D160C4}
    \definecolor{ansi-magenta-intense}{HTML}{A03196}
    \definecolor{ansi-cyan}{HTML}{60C6C8}
    \definecolor{ansi-cyan-intense}{HTML}{258F8F}
    \definecolor{ansi-white}{HTML}{C5C1B4}
    \definecolor{ansi-white-intense}{HTML}{A1A6B2}
    \definecolor{ansi-default-inverse-fg}{HTML}{FFFFFF}
    \definecolor{ansi-default-inverse-bg}{HTML}{000000}

    % common color for the border for error outputs.
    \definecolor{outerrorbackground}{HTML}{FFDFDF}

    % commands and environments needed by pandoc snippets
    % extracted from the output of `pandoc -s`
    \providecommand{\tightlist}{%
      \setlength{\itemsep}{0pt}\setlength{\parskip}{0pt}}
    \DefineVerbatimEnvironment{Highlighting}{Verbatim}{commandchars=\\\{\}}
    % Add ',fontsize=\small' for more characters per line
    \newenvironment{Shaded}{}{}
    \newcommand{\KeywordTok}[1]{\textcolor[rgb]{0.00,0.44,0.13}{\textbf{{#1}}}}
    \newcommand{\DataTypeTok}[1]{\textcolor[rgb]{0.56,0.13,0.00}{{#1}}}
    \newcommand{\DecValTok}[1]{\textcolor[rgb]{0.25,0.63,0.44}{{#1}}}
    \newcommand{\BaseNTok}[1]{\textcolor[rgb]{0.25,0.63,0.44}{{#1}}}
    \newcommand{\FloatTok}[1]{\textcolor[rgb]{0.25,0.63,0.44}{{#1}}}
    \newcommand{\CharTok}[1]{\textcolor[rgb]{0.25,0.44,0.63}{{#1}}}
    \newcommand{\StringTok}[1]{\textcolor[rgb]{0.25,0.44,0.63}{{#1}}}
    \newcommand{\CommentTok}[1]{\textcolor[rgb]{0.38,0.63,0.69}{\textit{{#1}}}}
    \newcommand{\OtherTok}[1]{\textcolor[rgb]{0.00,0.44,0.13}{{#1}}}
    \newcommand{\AlertTok}[1]{\textcolor[rgb]{1.00,0.00,0.00}{\textbf{{#1}}}}
    \newcommand{\FunctionTok}[1]{\textcolor[rgb]{0.02,0.16,0.49}{{#1}}}
    \newcommand{\RegionMarkerTok}[1]{{#1}}
    \newcommand{\ErrorTok}[1]{\textcolor[rgb]{1.00,0.00,0.00}{\textbf{{#1}}}}
    \newcommand{\NormalTok}[1]{{#1}}

    % Additional commands for more recent versions of Pandoc
    \newcommand{\ConstantTok}[1]{\textcolor[rgb]{0.53,0.00,0.00}{{#1}}}
    \newcommand{\SpecialCharTok}[1]{\textcolor[rgb]{0.25,0.44,0.63}{{#1}}}
    \newcommand{\VerbatimStringTok}[1]{\textcolor[rgb]{0.25,0.44,0.63}{{#1}}}
    \newcommand{\SpecialStringTok}[1]{\textcolor[rgb]{0.73,0.40,0.53}{{#1}}}
    \newcommand{\ImportTok}[1]{{#1}}
    \newcommand{\DocumentationTok}[1]{\textcolor[rgb]{0.73,0.13,0.13}{\textit{{#1}}}}
    \newcommand{\AnnotationTok}[1]{\textcolor[rgb]{0.38,0.63,0.69}{\textbf{\textit{{#1}}}}}
    \newcommand{\CommentVarTok}[1]{\textcolor[rgb]{0.38,0.63,0.69}{\textbf{\textit{{#1}}}}}
    \newcommand{\VariableTok}[1]{\textcolor[rgb]{0.10,0.09,0.49}{{#1}}}
    \newcommand{\ControlFlowTok}[1]{\textcolor[rgb]{0.00,0.44,0.13}{\textbf{{#1}}}}
    \newcommand{\OperatorTok}[1]{\textcolor[rgb]{0.40,0.40,0.40}{{#1}}}
    \newcommand{\BuiltInTok}[1]{{#1}}
    \newcommand{\ExtensionTok}[1]{{#1}}
    \newcommand{\PreprocessorTok}[1]{\textcolor[rgb]{0.74,0.48,0.00}{{#1}}}
    \newcommand{\AttributeTok}[1]{\textcolor[rgb]{0.49,0.56,0.16}{{#1}}}
    \newcommand{\InformationTok}[1]{\textcolor[rgb]{0.38,0.63,0.69}{\textbf{\textit{{#1}}}}}
    \newcommand{\WarningTok}[1]{\textcolor[rgb]{0.38,0.63,0.69}{\textbf{\textit{{#1}}}}}


    % Define a nice break command that doesn't care if a line doesn't already
    % exist.
    \def\br{\hspace*{\fill} \\* }
    % Math Jax compatibility definitions
    \def\gt{>}
    \def\lt{<}
    \let\Oldtex\TeX
    \let\Oldlatex\LaTeX
    \renewcommand{\TeX}{\textrm{\Oldtex}}
    \renewcommand{\LaTeX}{\textrm{\Oldlatex}}
    % Document parameters
    % Document title
    \title{101\_sympy\_dimentions\_manipulations}
    
    
    
    
    
    
    
% Pygments definitions
\makeatletter
\def\PY@reset{\let\PY@it=\relax \let\PY@bf=\relax%
    \let\PY@ul=\relax \let\PY@tc=\relax%
    \let\PY@bc=\relax \let\PY@ff=\relax}
\def\PY@tok#1{\csname PY@tok@#1\endcsname}
\def\PY@toks#1+{\ifx\relax#1\empty\else%
    \PY@tok{#1}\expandafter\PY@toks\fi}
\def\PY@do#1{\PY@bc{\PY@tc{\PY@ul{%
    \PY@it{\PY@bf{\PY@ff{#1}}}}}}}
\def\PY#1#2{\PY@reset\PY@toks#1+\relax+\PY@do{#2}}

\@namedef{PY@tok@w}{\def\PY@tc##1{\textcolor[rgb]{0.73,0.73,0.73}{##1}}}
\@namedef{PY@tok@c}{\let\PY@it=\textit\def\PY@tc##1{\textcolor[rgb]{0.24,0.48,0.48}{##1}}}
\@namedef{PY@tok@cp}{\def\PY@tc##1{\textcolor[rgb]{0.61,0.40,0.00}{##1}}}
\@namedef{PY@tok@k}{\let\PY@bf=\textbf\def\PY@tc##1{\textcolor[rgb]{0.00,0.50,0.00}{##1}}}
\@namedef{PY@tok@kp}{\def\PY@tc##1{\textcolor[rgb]{0.00,0.50,0.00}{##1}}}
\@namedef{PY@tok@kt}{\def\PY@tc##1{\textcolor[rgb]{0.69,0.00,0.25}{##1}}}
\@namedef{PY@tok@o}{\def\PY@tc##1{\textcolor[rgb]{0.40,0.40,0.40}{##1}}}
\@namedef{PY@tok@ow}{\let\PY@bf=\textbf\def\PY@tc##1{\textcolor[rgb]{0.67,0.13,1.00}{##1}}}
\@namedef{PY@tok@nb}{\def\PY@tc##1{\textcolor[rgb]{0.00,0.50,0.00}{##1}}}
\@namedef{PY@tok@nf}{\def\PY@tc##1{\textcolor[rgb]{0.00,0.00,1.00}{##1}}}
\@namedef{PY@tok@nc}{\let\PY@bf=\textbf\def\PY@tc##1{\textcolor[rgb]{0.00,0.00,1.00}{##1}}}
\@namedef{PY@tok@nn}{\let\PY@bf=\textbf\def\PY@tc##1{\textcolor[rgb]{0.00,0.00,1.00}{##1}}}
\@namedef{PY@tok@ne}{\let\PY@bf=\textbf\def\PY@tc##1{\textcolor[rgb]{0.80,0.25,0.22}{##1}}}
\@namedef{PY@tok@nv}{\def\PY@tc##1{\textcolor[rgb]{0.10,0.09,0.49}{##1}}}
\@namedef{PY@tok@no}{\def\PY@tc##1{\textcolor[rgb]{0.53,0.00,0.00}{##1}}}
\@namedef{PY@tok@nl}{\def\PY@tc##1{\textcolor[rgb]{0.46,0.46,0.00}{##1}}}
\@namedef{PY@tok@ni}{\let\PY@bf=\textbf\def\PY@tc##1{\textcolor[rgb]{0.44,0.44,0.44}{##1}}}
\@namedef{PY@tok@na}{\def\PY@tc##1{\textcolor[rgb]{0.41,0.47,0.13}{##1}}}
\@namedef{PY@tok@nt}{\let\PY@bf=\textbf\def\PY@tc##1{\textcolor[rgb]{0.00,0.50,0.00}{##1}}}
\@namedef{PY@tok@nd}{\def\PY@tc##1{\textcolor[rgb]{0.67,0.13,1.00}{##1}}}
\@namedef{PY@tok@s}{\def\PY@tc##1{\textcolor[rgb]{0.73,0.13,0.13}{##1}}}
\@namedef{PY@tok@sd}{\let\PY@it=\textit\def\PY@tc##1{\textcolor[rgb]{0.73,0.13,0.13}{##1}}}
\@namedef{PY@tok@si}{\let\PY@bf=\textbf\def\PY@tc##1{\textcolor[rgb]{0.64,0.35,0.47}{##1}}}
\@namedef{PY@tok@se}{\let\PY@bf=\textbf\def\PY@tc##1{\textcolor[rgb]{0.67,0.36,0.12}{##1}}}
\@namedef{PY@tok@sr}{\def\PY@tc##1{\textcolor[rgb]{0.64,0.35,0.47}{##1}}}
\@namedef{PY@tok@ss}{\def\PY@tc##1{\textcolor[rgb]{0.10,0.09,0.49}{##1}}}
\@namedef{PY@tok@sx}{\def\PY@tc##1{\textcolor[rgb]{0.00,0.50,0.00}{##1}}}
\@namedef{PY@tok@m}{\def\PY@tc##1{\textcolor[rgb]{0.40,0.40,0.40}{##1}}}
\@namedef{PY@tok@gh}{\let\PY@bf=\textbf\def\PY@tc##1{\textcolor[rgb]{0.00,0.00,0.50}{##1}}}
\@namedef{PY@tok@gu}{\let\PY@bf=\textbf\def\PY@tc##1{\textcolor[rgb]{0.50,0.00,0.50}{##1}}}
\@namedef{PY@tok@gd}{\def\PY@tc##1{\textcolor[rgb]{0.63,0.00,0.00}{##1}}}
\@namedef{PY@tok@gi}{\def\PY@tc##1{\textcolor[rgb]{0.00,0.52,0.00}{##1}}}
\@namedef{PY@tok@gr}{\def\PY@tc##1{\textcolor[rgb]{0.89,0.00,0.00}{##1}}}
\@namedef{PY@tok@ge}{\let\PY@it=\textit}
\@namedef{PY@tok@gs}{\let\PY@bf=\textbf}
\@namedef{PY@tok@ges}{\let\PY@bf=\textbf\let\PY@it=\textit}
\@namedef{PY@tok@gp}{\let\PY@bf=\textbf\def\PY@tc##1{\textcolor[rgb]{0.00,0.00,0.50}{##1}}}
\@namedef{PY@tok@go}{\def\PY@tc##1{\textcolor[rgb]{0.44,0.44,0.44}{##1}}}
\@namedef{PY@tok@gt}{\def\PY@tc##1{\textcolor[rgb]{0.00,0.27,0.87}{##1}}}
\@namedef{PY@tok@err}{\def\PY@bc##1{{\setlength{\fboxsep}{\string -\fboxrule}\fcolorbox[rgb]{1.00,0.00,0.00}{1,1,1}{\strut ##1}}}}
\@namedef{PY@tok@kc}{\let\PY@bf=\textbf\def\PY@tc##1{\textcolor[rgb]{0.00,0.50,0.00}{##1}}}
\@namedef{PY@tok@kd}{\let\PY@bf=\textbf\def\PY@tc##1{\textcolor[rgb]{0.00,0.50,0.00}{##1}}}
\@namedef{PY@tok@kn}{\let\PY@bf=\textbf\def\PY@tc##1{\textcolor[rgb]{0.00,0.50,0.00}{##1}}}
\@namedef{PY@tok@kr}{\let\PY@bf=\textbf\def\PY@tc##1{\textcolor[rgb]{0.00,0.50,0.00}{##1}}}
\@namedef{PY@tok@bp}{\def\PY@tc##1{\textcolor[rgb]{0.00,0.50,0.00}{##1}}}
\@namedef{PY@tok@fm}{\def\PY@tc##1{\textcolor[rgb]{0.00,0.00,1.00}{##1}}}
\@namedef{PY@tok@vc}{\def\PY@tc##1{\textcolor[rgb]{0.10,0.09,0.49}{##1}}}
\@namedef{PY@tok@vg}{\def\PY@tc##1{\textcolor[rgb]{0.10,0.09,0.49}{##1}}}
\@namedef{PY@tok@vi}{\def\PY@tc##1{\textcolor[rgb]{0.10,0.09,0.49}{##1}}}
\@namedef{PY@tok@vm}{\def\PY@tc##1{\textcolor[rgb]{0.10,0.09,0.49}{##1}}}
\@namedef{PY@tok@sa}{\def\PY@tc##1{\textcolor[rgb]{0.73,0.13,0.13}{##1}}}
\@namedef{PY@tok@sb}{\def\PY@tc##1{\textcolor[rgb]{0.73,0.13,0.13}{##1}}}
\@namedef{PY@tok@sc}{\def\PY@tc##1{\textcolor[rgb]{0.73,0.13,0.13}{##1}}}
\@namedef{PY@tok@dl}{\def\PY@tc##1{\textcolor[rgb]{0.73,0.13,0.13}{##1}}}
\@namedef{PY@tok@s2}{\def\PY@tc##1{\textcolor[rgb]{0.73,0.13,0.13}{##1}}}
\@namedef{PY@tok@sh}{\def\PY@tc##1{\textcolor[rgb]{0.73,0.13,0.13}{##1}}}
\@namedef{PY@tok@s1}{\def\PY@tc##1{\textcolor[rgb]{0.73,0.13,0.13}{##1}}}
\@namedef{PY@tok@mb}{\def\PY@tc##1{\textcolor[rgb]{0.40,0.40,0.40}{##1}}}
\@namedef{PY@tok@mf}{\def\PY@tc##1{\textcolor[rgb]{0.40,0.40,0.40}{##1}}}
\@namedef{PY@tok@mh}{\def\PY@tc##1{\textcolor[rgb]{0.40,0.40,0.40}{##1}}}
\@namedef{PY@tok@mi}{\def\PY@tc##1{\textcolor[rgb]{0.40,0.40,0.40}{##1}}}
\@namedef{PY@tok@il}{\def\PY@tc##1{\textcolor[rgb]{0.40,0.40,0.40}{##1}}}
\@namedef{PY@tok@mo}{\def\PY@tc##1{\textcolor[rgb]{0.40,0.40,0.40}{##1}}}
\@namedef{PY@tok@ch}{\let\PY@it=\textit\def\PY@tc##1{\textcolor[rgb]{0.24,0.48,0.48}{##1}}}
\@namedef{PY@tok@cm}{\let\PY@it=\textit\def\PY@tc##1{\textcolor[rgb]{0.24,0.48,0.48}{##1}}}
\@namedef{PY@tok@cpf}{\let\PY@it=\textit\def\PY@tc##1{\textcolor[rgb]{0.24,0.48,0.48}{##1}}}
\@namedef{PY@tok@c1}{\let\PY@it=\textit\def\PY@tc##1{\textcolor[rgb]{0.24,0.48,0.48}{##1}}}
\@namedef{PY@tok@cs}{\let\PY@it=\textit\def\PY@tc##1{\textcolor[rgb]{0.24,0.48,0.48}{##1}}}

\def\PYZbs{\char`\\}
\def\PYZus{\char`\_}
\def\PYZob{\char`\{}
\def\PYZcb{\char`\}}
\def\PYZca{\char`\^}
\def\PYZam{\char`\&}
\def\PYZlt{\char`\<}
\def\PYZgt{\char`\>}
\def\PYZsh{\char`\#}
\def\PYZpc{\char`\%}
\def\PYZdl{\char`\$}
\def\PYZhy{\char`\-}
\def\PYZsq{\char`\'}
\def\PYZdq{\char`\"}
\def\PYZti{\char`\~}
% for compatibility with earlier versions
\def\PYZat{@}
\def\PYZlb{[}
\def\PYZrb{]}
\makeatother


    % For linebreaks inside Verbatim environment from package fancyvrb.
    \makeatletter
        \newbox\Wrappedcontinuationbox
        \newbox\Wrappedvisiblespacebox
        \newcommand*\Wrappedvisiblespace {\textcolor{red}{\textvisiblespace}}
        \newcommand*\Wrappedcontinuationsymbol {\textcolor{red}{\llap{\tiny$\m@th\hookrightarrow$}}}
        \newcommand*\Wrappedcontinuationindent {3ex }
        \newcommand*\Wrappedafterbreak {\kern\Wrappedcontinuationindent\copy\Wrappedcontinuationbox}
        % Take advantage of the already applied Pygments mark-up to insert
        % potential linebreaks for TeX processing.
        %        {, <, #, %, $, ' and ": go to next line.
        %        _, }, ^, &, >, - and ~: stay at end of broken line.
        % Use of \textquotesingle for straight quote.
        \newcommand*\Wrappedbreaksatspecials {%
            \def\PYGZus{\discretionary{\char`\_}{\Wrappedafterbreak}{\char`\_}}%
            \def\PYGZob{\discretionary{}{\Wrappedafterbreak\char`\{}{\char`\{}}%
            \def\PYGZcb{\discretionary{\char`\}}{\Wrappedafterbreak}{\char`\}}}%
            \def\PYGZca{\discretionary{\char`\^}{\Wrappedafterbreak}{\char`\^}}%
            \def\PYGZam{\discretionary{\char`\&}{\Wrappedafterbreak}{\char`\&}}%
            \def\PYGZlt{\discretionary{}{\Wrappedafterbreak\char`\<}{\char`\<}}%
            \def\PYGZgt{\discretionary{\char`\>}{\Wrappedafterbreak}{\char`\>}}%
            \def\PYGZsh{\discretionary{}{\Wrappedafterbreak\char`\#}{\char`\#}}%
            \def\PYGZpc{\discretionary{}{\Wrappedafterbreak\char`\%}{\char`\%}}%
            \def\PYGZdl{\discretionary{}{\Wrappedafterbreak\char`\$}{\char`\$}}%
            \def\PYGZhy{\discretionary{\char`\-}{\Wrappedafterbreak}{\char`\-}}%
            \def\PYGZsq{\discretionary{}{\Wrappedafterbreak\textquotesingle}{\textquotesingle}}%
            \def\PYGZdq{\discretionary{}{\Wrappedafterbreak\char`\"}{\char`\"}}%
            \def\PYGZti{\discretionary{\char`\~}{\Wrappedafterbreak}{\char`\~}}%
        }
        % Some characters . , ; ? ! / are not pygmentized.
        % This macro makes them "active" and they will insert potential linebreaks
        \newcommand*\Wrappedbreaksatpunct {%
            \lccode`\~`\.\lowercase{\def~}{\discretionary{\hbox{\char`\.}}{\Wrappedafterbreak}{\hbox{\char`\.}}}%
            \lccode`\~`\,\lowercase{\def~}{\discretionary{\hbox{\char`\,}}{\Wrappedafterbreak}{\hbox{\char`\,}}}%
            \lccode`\~`\;\lowercase{\def~}{\discretionary{\hbox{\char`\;}}{\Wrappedafterbreak}{\hbox{\char`\;}}}%
            \lccode`\~`\:\lowercase{\def~}{\discretionary{\hbox{\char`\:}}{\Wrappedafterbreak}{\hbox{\char`\:}}}%
            \lccode`\~`\?\lowercase{\def~}{\discretionary{\hbox{\char`\?}}{\Wrappedafterbreak}{\hbox{\char`\?}}}%
            \lccode`\~`\!\lowercase{\def~}{\discretionary{\hbox{\char`\!}}{\Wrappedafterbreak}{\hbox{\char`\!}}}%
            \lccode`\~`\/\lowercase{\def~}{\discretionary{\hbox{\char`\/}}{\Wrappedafterbreak}{\hbox{\char`\/}}}%
            \catcode`\.\active
            \catcode`\,\active
            \catcode`\;\active
            \catcode`\:\active
            \catcode`\?\active
            \catcode`\!\active
            \catcode`\/\active
            \lccode`\~`\~
        }
    \makeatother

    \let\OriginalVerbatim=\Verbatim
    \makeatletter
    \renewcommand{\Verbatim}[1][1]{%
        %\parskip\z@skip
        \sbox\Wrappedcontinuationbox {\Wrappedcontinuationsymbol}%
        \sbox\Wrappedvisiblespacebox {\FV@SetupFont\Wrappedvisiblespace}%
        \def\FancyVerbFormatLine ##1{\hsize\linewidth
            \vtop{\raggedright\hyphenpenalty\z@\exhyphenpenalty\z@
                \doublehyphendemerits\z@\finalhyphendemerits\z@
                \strut ##1\strut}%
        }%
        % If the linebreak is at a space, the latter will be displayed as visible
        % space at end of first line, and a continuation symbol starts next line.
        % Stretch/shrink are however usually zero for typewriter font.
        \def\FV@Space {%
            \nobreak\hskip\z@ plus\fontdimen3\font minus\fontdimen4\font
            \discretionary{\copy\Wrappedvisiblespacebox}{\Wrappedafterbreak}
            {\kern\fontdimen2\font}%
        }%

        % Allow breaks at special characters using \PYG... macros.
        \Wrappedbreaksatspecials
        % Breaks at punctuation characters . , ; ? ! and / need catcode=\active
        \OriginalVerbatim[#1,codes*=\Wrappedbreaksatpunct]%
    }
    \makeatother

    % Exact colors from NB
    \definecolor{incolor}{HTML}{303F9F}
    \definecolor{outcolor}{HTML}{D84315}
    \definecolor{cellborder}{HTML}{CFCFCF}
    \definecolor{cellbackground}{HTML}{F7F7F7}

    % prompt
    \makeatletter
    \newcommand{\boxspacing}{\kern\kvtcb@left@rule\kern\kvtcb@boxsep}
    \makeatother
    \newcommand{\prompt}[4]{
        {\ttfamily\llap{{\color{#2}[#3]:\hspace{3pt}#4}}\vspace{-\baselineskip}}
    }
    

    
    % Prevent overflowing lines due to hard-to-break entities
    \sloppy
    % Setup hyperref package
    \hypersetup{
      breaklinks=true,  % so long urls are correctly broken across lines
      colorlinks=true,
      urlcolor=urlcolor,
      linkcolor=linkcolor,
      citecolor=citecolor,
      }
    % Slightly bigger margins than the latex defaults
    
    \geometry{verbose,tmargin=1in,bmargin=1in,lmargin=1in,rmargin=1in}
    
    

\begin{document}
    
    \maketitle
    
    

    
    \hypertarget{ux43fux440ux438ux43cux435ux440ux44b-ux43fux440ux435ux43eux431ux440ux430ux437ux43eux432ux430ux43dux438ux44f-ux440ux430ux437ux43cux435ux440ux43dux44bux445-ux432ux435ux43bux438ux447ux438ux43d}{%
\section{Примеры преобразования размерных
величин}\label{ux43fux440ux438ux43cux435ux440ux44b-ux43fux440ux435ux43eux431ux440ux430ux437ux43eux432ux430ux43dux438ux44f-ux440ux430ux437ux43cux435ux440ux43dux44bux445-ux432ux435ux43bux438ux447ux438ux43d}}

Преобразования размерных величин удобно выполнять с модулем символьных
вычислений \texttt{python} - \texttt{sympy}. Преобразования размерностей
ключевых величин полезно знать наизусть, хотя всегда можно найти их в
таблицах. Значения многих физические константы зашины в модуле
\texttt{scipy.constants}, иногда это оказывается удобным, при этом
автоматически будет учитываться достаточно большое количество знаков
после запятой в константах. Рассмотрим размерности ряда величин широко
применяемых в нефтяном инжиниринге.

    \begin{tcolorbox}[breakable, size=fbox, boxrule=1pt, pad at break*=1mm,colback=cellbackground, colframe=cellborder]
\prompt{In}{incolor}{1}{\boxspacing}
\begin{Verbatim}[commandchars=\\\{\}]
\PY{k+kn}{import} \PY{n+nn}{sympy} \PY{k}{as} \PY{n+nn}{sp}
\PY{k+kn}{import} \PY{n+nn}{scipy}\PY{n+nn}{.}\PY{n+nn}{constants} \PY{k}{as} \PY{n+nn}{const}
\end{Verbatim}
\end{tcolorbox}

    \begin{tcolorbox}[breakable, size=fbox, boxrule=1pt, pad at break*=1mm,colback=cellbackground, colframe=cellborder]
\prompt{In}{incolor}{2}{\boxspacing}
\begin{Verbatim}[commandchars=\\\{\}]
\PY{c+c1}{\PYZsh{} в модуле scipy.constants есть значения общепринятых констант \PYZhy{}например значение pi}
\PY{n+nb}{print}\PY{p}{(}\PY{n}{const}\PY{o}{.}\PY{n}{pi}\PY{p}{)}
\end{Verbatim}
\end{tcolorbox}

    \begin{Verbatim}[commandchars=\\\{\}]
3.141592653589793
    \end{Verbatim}

    \hypertarget{ux43eux431ux44aux435ux43cux43dux44bux439-ux440ux430ux441ux445ux43eux434-q}{%
\subsubsection{\texorpdfstring{Объемный расход
\(q\)}{Объемный расход q}}\label{ux43eux431ux44aux435ux43cux43dux44bux439-ux440ux430ux441ux445ux43eux434-q}}

В СИ измеряется в {[}м\(^3\)/сек{]}, в практических метрических единицах
измеряется в {[}м\(^3\)/сут{]}, в американских промысловых единицах
измеряется в {[}bbl/day{]}.

\begin{itemize}
\tightlist
\item
  \(1\) {[}м\(^3\)/сек{]} = \(543439\) {[}bbl/day{]} = \(86400\)
  {[}м\(^3\)/сут{]}
\item
  \(1\) {[}м\(^3\)/сут{]} = \(\dfrac{1}{86400}\) {[}м\(^3\)/сек{]}
  \(= 1.157407 \cdot 10^{-5}\) {[}м\(^3\)/сек{]}
\item
  \(1\) {[}bbl/day{]} = \(0.15898\) {[}м\(^3\)/сут{]}
\end{itemize}

    \begin{tcolorbox}[breakable, size=fbox, boxrule=1pt, pad at break*=1mm,colback=cellbackground, colframe=cellborder]
\prompt{In}{incolor}{8}{\boxspacing}
\begin{Verbatim}[commandchars=\\\{\}]
\PY{c+c1}{\PYZsh{} выведем некоторые переводные коэффициенты для объемных расходов}
\PY{n+nb}{print}\PY{p}{(}\PY{l+s+sa}{f}\PY{l+s+s1}{\PYZsq{}}\PY{l+s+s1}{Одни [сут] = }\PY{l+s+si}{\PYZob{}}\PY{l+m+mi}{24}\PY{o}{*}\PY{l+m+mi}{60}\PY{o}{*}\PY{l+m+mi}{60}\PY{l+s+si}{\PYZcb{}}\PY{l+s+s1}{ = }\PY{l+s+si}{\PYZob{}}\PY{n}{const}\PY{o}{.}\PY{n}{day}\PY{l+s+si}{\PYZcb{}}\PY{l+s+s1}{  [сек]}\PY{l+s+s1}{\PYZsq{}}\PY{p}{)}
\PY{n+nb}{print}\PY{p}{(}\PY{l+s+sa}{f}\PY{l+s+s1}{\PYZsq{}}\PY{l+s+s1}{Один [м3/сут] = }\PY{l+s+si}{\PYZob{}}\PY{l+m+mi}{1}\PY{o}{/}\PY{n}{const}\PY{o}{.}\PY{n}{day}\PY{l+s+si}{\PYZcb{}}\PY{l+s+s1}{ [м3/сек]}\PY{l+s+s1}{\PYZsq{}}\PY{p}{)}
\PY{n+nb}{print}\PY{p}{(}\PY{l+s+sa}{f}\PY{l+s+s1}{\PYZsq{}}\PY{l+s+s1}{Один баррель в день [bbl/day] = }\PY{l+s+si}{\PYZob{}}\PY{n}{const}\PY{o}{.}\PY{n}{bbl}\PY{l+s+si}{\PYZcb{}}\PY{l+s+s1}{ [м3/сут]}\PY{l+s+s1}{\PYZsq{}}\PY{p}{)}
\PY{n+nb}{print}\PY{p}{(}\PY{l+s+sa}{f}\PY{l+s+s1}{\PYZsq{}}\PY{l+s+s1}{Один баррель в день [bbl/day] = }\PY{l+s+si}{\PYZob{}}\PY{n}{const}\PY{o}{.}\PY{n}{bbl}\PY{o}{/}\PY{n}{const}\PY{o}{.}\PY{n}{day}\PY{l+s+si}{\PYZcb{}}\PY{l+s+s1}{ [м3/сек]}\PY{l+s+s1}{\PYZsq{}}\PY{p}{)}
\PY{n+nb}{print}\PY{p}{(}\PY{l+s+sa}{f}\PY{l+s+s1}{\PYZsq{}}\PY{l+s+s1}{Один [м3/сут] = }\PY{l+s+si}{\PYZob{}}\PY{l+m+mi}{1}\PY{o}{/}\PY{n}{const}\PY{o}{.}\PY{n}{day}\PY{l+s+si}{\PYZcb{}}\PY{l+s+s1}{ [м3/сек]}\PY{l+s+s1}{\PYZsq{}}\PY{p}{)}
\end{Verbatim}
\end{tcolorbox}

    \begin{Verbatim}[commandchars=\\\{\}]
Одни [сут] = 86400 = 86400.0  [сек]
Один [м3/сут] = 1.1574074074074073e-05 [м3/сек]
Один баррель в день [bbl/day] = 0.15898729492799998 [м3/сут]
Один баррель в день [bbl/day] = 1.8401307283333331e-06 [м3/сек]
Один [м3/сут] = 1.1574074074074073e-05 [м3/сек]
    \end{Verbatim}

    \hypertarget{ux43fux440ux43eux43dux438ux446ux430ux435ux43cux43eux441ux442ux44c-k}{%
\subsubsection{\texorpdfstring{Проницаемость
\(k\)}{Проницаемость k}}\label{ux43fux440ux43eux43dux438ux446ux430ux435ux43cux43eux441ux442ux44c-k}}

В СИ измеряется в {[}м\(^2\){]}, в практических метрических единицах
измеряется в {[}мД{]}, в американских промысловых единицах измеряется в
{[}mD{]}.

Определение: в пористой среде с проницаемостью в один Дарси для
поддержания течения жидкости с динамической вязкостью 1 сП со скоростью
фильтрации 1 см/с необходимо поддерживать перепад давления жидкости
приблизительно в одну атмосферу на 1 см вдоль направления течения. При
использовании физической атмосферы для расчета перепада давления
(физическая атмосфера = 101 325 Па) единица проницаемости равняется
приблизительно 0.986923 мкм².

В отечественной литературе при определении дарси в качестве величины
атмосферы было принято использовать техническую атмосферу (1 кгс/см² =
98 066,5 Па), так что для величины дарси получалось значение
приблизительно 1,02 мкм², причём эпизодические случаи использования
западного определения дарси специально отмечались
{[}\href{https://ru.wikipedia.org/wiki/\%D0\%94\%D0\%B0\%D1\%80\%D1\%81\%D0\%B8}{ru.wikipedia.org/wiki/Дарси}{]}.
Согласно ГОСТ 26450.2-85 величины 1 Дарси \(= 0.9869⋅10^{−12}\) м².

\begin{itemize}
\tightlist
\item
  \(1\) {[}м\(^2\){]} = \(1.01325 \cdot 10^{15}\) {[}мД{]}
\item
  \(1\) {[}мД{]} = \(0.986923 \cdot 10^{-15}\) {[}м\(^2\){]}
\end{itemize}

    \begin{tcolorbox}[breakable, size=fbox, boxrule=1pt, pad at break*=1mm,colback=cellbackground, colframe=cellborder]
\prompt{In}{incolor}{9}{\boxspacing}
\begin{Verbatim}[commandchars=\\\{\}]
\PY{n+nb}{print}\PY{p}{(}\PY{l+s+sa}{f}\PY{l+s+s1}{\PYZsq{}}\PY{l+s+s1}{Один [мД] = }\PY{l+s+si}{\PYZob{}}\PY{l+m+mf}{1e5}\PY{o}{/}\PY{n}{const}\PY{o}{.}\PY{n}{atm}\PY{+w}{ }\PY{o}{*}\PY{+w}{ }\PY{l+m+mf}{1e\PYZhy{}15}\PY{l+s+si}{\PYZcb{}}\PY{l+s+s1}{ [м²]}\PY{l+s+s1}{\PYZsq{}}\PY{p}{)}
\end{Verbatim}
\end{tcolorbox}

    \begin{Verbatim}[commandchars=\\\{\}]
Один [мД] = 9.86923266716013e-16 [м²]
    \end{Verbatim}

    \hypertarget{ux432ux44fux437ux43aux43eux441ux442ux44c-mu}{%
\subsubsection{\texorpdfstring{Вязкость
\(\mu\)}{Вязкость \textbackslash mu}}\label{ux432ux44fux437ux43aux43eux441ux442ux44c-mu}}

\begin{itemize}
\tightlist
\item
  \(1\) {[}Па\(\cdot\)с{]} = \(1000\) {[}сП{]}
\item
  \(1\) {[}сП{]} = \(10^{-3}\) {[}Па\(\cdot\)с{]}
\end{itemize}

\hypertarget{ux434ux430ux432ux43bux435ux43dux438ux435-p}{%
\subsubsection{\texorpdfstring{Давление
\(p\)}{Давление p}}\label{ux434ux430ux432ux43bux435ux43dux438ux435-p}}

\begin{itemize}
\tightlist
\item
  \(1\) {[}Па{]} = \(0.0001450\) {[}psi{]} = \(0.00000987\) {[}атм{]}
\item
  \(1\) {[}атм{]} = \(14.6959\) {[}psi{]} = \(101325\) {[}Па{]}
\end{itemize}

    \begin{tcolorbox}[breakable, size=fbox, boxrule=1pt, pad at break*=1mm,colback=cellbackground, colframe=cellborder]
\prompt{In}{incolor}{5}{\boxspacing}
\begin{Verbatim}[commandchars=\\\{\}]
\PY{n}{AT} \PY{o}{=} \PY{l+m+mf}{98066.5}  \PY{c+c1}{\PYZsh{} technical atmosphere in Pa,  техническая атмосфера в Па}
\PY{n+nb}{print}\PY{p}{(}\PY{l+s+sa}{f}\PY{l+s+s1}{\PYZsq{}}\PY{l+s+s1}{Один  [psi] в [Па] = }\PY{l+s+si}{\PYZob{}}\PY{n}{const}\PY{o}{.}\PY{n}{psi}\PY{l+s+si}{\PYZcb{}}\PY{l+s+s1}{\PYZsq{}}\PY{p}{)}
\PY{n+nb}{print}\PY{p}{(}\PY{l+s+sa}{f}\PY{l+s+s1}{\PYZsq{}}\PY{l+s+s1}{Один  [bar] в [Па] = }\PY{l+s+si}{\PYZob{}}\PY{n}{const}\PY{o}{.}\PY{n}{bar}\PY{l+s+si}{\PYZcb{}}\PY{l+s+s1}{\PYZsq{}}\PY{p}{)}
\PY{n+nb}{print}\PY{p}{(}\PY{l+s+sa}{f}\PY{l+s+s1}{\PYZsq{}}\PY{l+s+s1}{Один  [atm] в [Па] = }\PY{l+s+si}{\PYZob{}}\PY{n}{const}\PY{o}{.}\PY{n}{atm}\PY{l+s+si}{\PYZcb{}}\PY{l+s+s1}{\PYZsq{}}\PY{p}{)}
\PY{n+nb}{print}\PY{p}{(}\PY{l+s+sa}{f}\PY{l+s+s1}{\PYZsq{}}\PY{l+s+s1}{Один  [at] в [Па] = }\PY{l+s+si}{\PYZob{}}\PY{n}{AT}\PY{l+s+si}{\PYZcb{}}\PY{l+s+s1}{\PYZsq{}}\PY{p}{)}
\PY{n+nb}{print}\PY{p}{(}\PY{l+s+sa}{f}\PY{l+s+s1}{\PYZsq{}}\PY{l+s+s1}{Один  [atm] в [psi] = }\PY{l+s+si}{\PYZob{}}\PY{n}{const}\PY{o}{.}\PY{n}{atm}\PY{o}{/}\PY{n}{const}\PY{o}{.}\PY{n}{psi}\PY{l+s+si}{\PYZcb{}}\PY{l+s+s1}{\PYZsq{}}\PY{p}{)}
\end{Verbatim}
\end{tcolorbox}

    \begin{Verbatim}[commandchars=\\\{\}]
Один  [psi] в [Па] = 6894.757293168361
Один  [bar] в [Па] = 100000.0
Один  [atm] в [Па] = 101325.0
Один  [at] в [Па] = 98066.5
Один  [atm] в [psi] = 14.69594877551345
    \end{Verbatim}

    \hypertarget{ux440ux430ux441ux441ux442ux43eux44fux43dux438ux435-x}{%
\subsubsection{\texorpdfstring{Расстояние
\(x\)}{Расстояние x}}\label{ux440ux430ux441ux441ux442ux43eux44fux43dux438ux435-x}}

\begin{itemize}
\tightlist
\item
  \(1\) {[}м{]} = \(3.28\) {[}ft{]}
\end{itemize}

    \hypertarget{ux440ux430ux437ux43cux435ux440ux43dux44bux439-ux43aux43eux44dux444ux444ux438ux446ux438ux435ux43dux442-ux434ux43bux44f-ux444ux43eux440ux43cux443ux43bux44b-ux434ux44eux43fux44eux438}{%
\section{Размерный коэффициент для формулы
Дюпюи}\label{ux440ux430ux437ux43cux435ux440ux43dux44bux439-ux43aux43eux44dux444ux444ux438ux446ux438ux435ux43dux442-ux434ux43bux44f-ux444ux43eux440ux43cux443ux43bux44b-ux434ux44eux43fux44eux438}}

Используя рассчитанные выше переводные коэффициенты для различных
размерных величин рассчитаем переводной коэффициент в формуле Дюпюи

\[ Q = \frac{ 2 \pi kh}{ \mu B} \frac{  \left( p_i - p \right) } {\ln{\dfrac{r_e}{r_w}} +S } \]

    \begin{tcolorbox}[breakable, size=fbox, boxrule=1pt, pad at break*=1mm,colback=cellbackground, colframe=cellborder]
\prompt{In}{incolor}{23}{\boxspacing}
\begin{Verbatim}[commandchars=\\\{\}]
\PY{c+c1}{\PYZsh{} зададим переменные sympy}
\PY{n}{Q}\PY{p}{,} \PY{n}{k}\PY{p}{,} \PY{n}{h}\PY{p}{,} \PY{n}{mu}\PY{p}{,} \PY{n}{B}\PY{p}{,} \PY{n}{pres}\PY{p}{,} \PY{n}{pwf}\PY{p}{,} \PY{n}{re}\PY{p}{,} \PY{n}{rw}\PY{p}{,} \PY{n}{S}\PY{p}{,} \PY{n}{pi} \PY{o}{=} \PY{n}{sp}\PY{o}{.}\PY{n}{symbols}\PY{p}{(}\PY{l+s+s1}{\PYZsq{}}\PY{l+s+s1}{Q k h mu B p\PYZus{}res p\PYZus{}wf r\PYZus{}e r\PYZus{}w S pi}\PY{l+s+s1}{\PYZsq{}}\PY{p}{,} \PY{n}{real}\PY{o}{=}\PY{k+kc}{True}\PY{p}{,} \PY{n}{positive}\PY{o}{=}\PY{k+kc}{True}\PY{p}{)}
\end{Verbatim}
\end{tcolorbox}

    \begin{tcolorbox}[breakable, size=fbox, boxrule=1pt, pad at break*=1mm,colback=cellbackground, colframe=cellborder]
\prompt{In}{incolor}{24}{\boxspacing}
\begin{Verbatim}[commandchars=\\\{\}]
\PY{c+c1}{\PYZsh{} определим уравнение}
\PY{n}{eq} \PY{o}{=} \PY{n}{sp}\PY{o}{.}\PY{n}{Eq}\PY{p}{(}\PY{n}{Q}\PY{p}{,} \PY{l+m+mi}{2} \PY{o}{*} \PY{n}{pi} \PY{o}{*} \PY{n}{k} \PY{o}{*} \PY{n}{h} \PY{o}{/} \PY{p}{(}\PY{n}{mu} \PY{o}{*} \PY{n}{B}\PY{p}{)} \PY{o}{*} \PY{p}{(}\PY{n}{pres} \PY{o}{\PYZhy{}} \PY{n}{pwf}\PY{p}{)} \PY{o}{/} \PY{p}{(}\PY{n}{sp}\PY{o}{.}\PY{n}{ln}\PY{p}{(}\PY{n}{re}\PY{o}{/}\PY{n}{rw}\PY{p}{)} \PY{o}{+} \PY{n}{S}\PY{p}{)}\PY{p}{)}
\PY{n}{eq}
\end{Verbatim}
\end{tcolorbox}
 
            
\prompt{Out}{outcolor}{24}{}
    
    $\displaystyle Q = \frac{2 h k \pi \left(p_{res} - p_{wf}\right)}{B \mu \left(S + \log{\left(\frac{r_{e}}{r_{w}} \right)}\right)}$

    

    \begin{tcolorbox}[breakable, size=fbox, boxrule=1pt, pad at break*=1mm,colback=cellbackground, colframe=cellborder]
\prompt{In}{incolor}{25}{\boxspacing}
\begin{Verbatim}[commandchars=\\\{\}]
\PY{c+c1}{\PYZsh{} подставим в уравнение переводные величины}
\PY{n}{eq} \PY{o}{=} \PY{n}{eq}\PY{o}{.}\PY{n}{subs}\PY{p}{(}\PY{n}{Q}\PY{p}{,} \PY{l+m+mi}{1}\PY{o}{/}\PY{n}{const}\PY{o}{.}\PY{n}{day} \PY{o}{*} \PY{n}{Q}\PY{p}{)}  \PY{c+c1}{\PYZsh{} дебит, [м3/сут] в [м3/сек]}
\PY{n}{f\PYZus{}k} \PY{o}{=} \PY{l+m+mf}{1e5}\PY{o}{/}\PY{n}{const}\PY{o}{.}\PY{n}{atm} \PY{o}{*} \PY{l+m+mf}{1e\PYZhy{}15} 
\PY{n}{eq} \PY{o}{=} \PY{n}{eq}\PY{o}{.}\PY{n}{subs}\PY{p}{(}\PY{n}{k}\PY{p}{,} \PY{n}{f\PYZus{}k} \PY{o}{*} \PY{n}{k}\PY{p}{)}     \PY{c+c1}{\PYZsh{} проницаемость, [мД] в [м2]}
\PY{n}{eq} \PY{o}{=} \PY{n}{eq}\PY{o}{.}\PY{n}{subs}\PY{p}{(}\PY{n}{mu}\PY{p}{,} \PY{l+m+mf}{1e\PYZhy{}3} \PY{o}{*} \PY{n}{mu}\PY{p}{)}  \PY{c+c1}{\PYZsh{} вязкость, [сП] в [Па сек]}
\PY{n}{eq} \PY{o}{=} \PY{n}{eq}\PY{o}{.}\PY{n}{subs}\PY{p}{(}\PY{n}{pres}\PY{p}{,} \PY{n}{const}\PY{o}{.}\PY{n}{atm} \PY{o}{*} \PY{n}{pres}\PY{p}{)} \PY{c+c1}{\PYZsh{} давление [атм] в [Па]}
\PY{n}{eq} \PY{o}{=} \PY{n}{eq}\PY{o}{.}\PY{n}{subs}\PY{p}{(}\PY{n}{pwf}\PY{p}{,} \PY{n}{const}\PY{o}{.}\PY{n}{atm} \PY{o}{*} \PY{n}{pwf}\PY{p}{)} \PY{c+c1}{\PYZsh{} давление [атм] в [Па]}
\PY{n}{eq} \PY{o}{=} \PY{n}{eq}\PY{o}{.}\PY{n}{subs}\PY{p}{(}\PY{n}{pi}\PY{p}{,} \PY{n}{const}\PY{o}{.}\PY{n}{pi}\PY{p}{)}

\PY{n}{display}\PY{p}{(}\PY{n}{eq}\PY{p}{)}
\end{Verbatim}
\end{tcolorbox}

    $\displaystyle 1.15740740740741 \cdot 10^{-5} Q = \frac{6.20102176874373 \cdot 10^{-12} h k \left(101325.0 p_{res} - 101325.0 p_{wf}\right)}{B \mu \left(S + \log{\left(\frac{r_{e}}{r_{w}} \right)}\right)}$

    
    Решим полученное уравнение относительно Q и упростим средствами
\texttt{sympy}

    \begin{tcolorbox}[breakable, size=fbox, boxrule=1pt, pad at break*=1mm,colback=cellbackground, colframe=cellborder]
\prompt{In}{incolor}{26}{\boxspacing}
\begin{Verbatim}[commandchars=\\\{\}]
\PY{n}{eq1} \PY{o}{=} \PY{n}{sp}\PY{o}{.}\PY{n}{simplify}\PY{p}{(}\PY{n}{sp}\PY{o}{.}\PY{n}{solve}\PY{p}{(}\PY{n}{eq}\PY{p}{,}\PY{n}{Q}\PY{p}{)}\PY{p}{[}\PY{l+m+mi}{0}\PY{p}{]}\PY{p}{)}
\PY{n}{display}\PY{p}{(}\PY{n}{sp}\PY{o}{.}\PY{n}{Eq}\PY{p}{(}\PY{n}{Q}\PY{p}{,}\PY{n}{eq1}\PY{p}{)}\PY{p}{)}
\end{Verbatim}
\end{tcolorbox}

    $\displaystyle Q = \frac{0.0542867210540316 h k \left(p_{res} - p_{wf}\right)}{B \mu \left(S + \log{\left(\frac{r_{e}}{r_{w}} \right)}\right)}$

    
    Выделим полученную константу в явном виде и найдем обратную величину -
это и будет необходимый нам переводной коэффициент.

    \begin{tcolorbox}[breakable, size=fbox, boxrule=1pt, pad at break*=1mm,colback=cellbackground, colframe=cellborder]
\prompt{In}{incolor}{27}{\boxspacing}
\begin{Verbatim}[commandchars=\\\{\}]
\PY{n}{f} \PY{o}{=} \PY{l+m+mi}{1}\PY{o}{/}\PY{n}{eq1}\PY{o}{.}\PY{n}{args}\PY{p}{[}\PY{l+m+mi}{0}\PY{p}{]}
\PY{n}{f}
\end{Verbatim}
\end{tcolorbox}
 
            
\prompt{Out}{outcolor}{27}{}
    
    $\displaystyle 18.4207110060064$

    

    По умолчанию \texttt{sympy} автоматически организует порядок элементов в
своих выражениях. Этот порядок может отличаться от привычного - хотя и
суть формул при этом не меняется. Применяя некоторые хитрости можно
заставить \texttt{sympy} вывести выражения в приемлимом виде.

    \begin{tcolorbox}[breakable, size=fbox, boxrule=1pt, pad at break*=1mm,colback=cellbackground, colframe=cellborder]
\prompt{In}{incolor}{28}{\boxspacing}
\begin{Verbatim}[commandchars=\\\{\}]
\PY{n}{a} \PY{o}{=} \PY{n}{sp}\PY{o}{.}\PY{n}{symbols}\PY{p}{(}\PY{l+s+s1}{\PYZsq{}}\PY{l+s+s1}{a}\PY{l+s+s1}{\PYZsq{}}\PY{p}{)}
\PY{n}{eq2} \PY{o}{=} \PY{n}{eq1}\PY{o}{.}\PY{n}{subs}\PY{p}{(}\PY{n}{eq1}\PY{o}{.}\PY{n}{args}\PY{p}{[}\PY{l+m+mi}{0}\PY{p}{]}\PY{p}{,}\PY{l+m+mi}{1}\PY{o}{/}\PY{n}{a}\PY{p}{)}
\PY{k}{with} \PY{n}{sp}\PY{o}{.}\PY{n}{evaluate}\PY{p}{(}\PY{k+kc}{False}\PY{p}{)}\PY{p}{:}
    \PY{n}{display}\PY{p}{(}\PY{n}{eq2}\PY{o}{.}\PY{n}{subs}\PY{p}{(}\PY{n}{a}\PY{p}{,} \PY{n}{f}\PY{p}{)}\PY{p}{)}
\end{Verbatim}
\end{tcolorbox}

    $\displaystyle \frac{h k \left(p_{res} - p_{wf}\right)}{18.4207110060064 B \mu \left(S + \log{\left(\frac{r_{e}}{r_{w}} \right)}\right)}$

    
    Но иногда результат проще переписать руками в нужном виде. В итоге
уравнение Дюпюи в практических метрических единицах измерения примет
вид.

\[ Q = \frac{ kh}{18.42  \mu B} \frac{  \left( p_i - p \right) } {\ln{\dfrac{r_e}{r_w}} +S } \]

где

\begin{itemize}
\tightlist
\item
  \(Q\) - дебит скважины на поверхности, приведенный к нормальным
  условиям, ст. м\(^3\)/сут
\item
  \(\mu\) - вязкость нефти в пласте, сП
\item
  \(B\) - объемный коэффициент нефти, м\(^3\)/м\(^3\)
\item
  \(P_{res}\) - пластовое давление или давление на контуре с радиусом
  \(r_e\), атма
\item
  \(P_{wf}\) - давление забойное, атма
\item
  \(k\) - проницаемость, мД
\item
  \(h\) - мощность пласта, м
\item
  \(r_e\) - внешний контур дренирования скважины, м
\item
  \(r_w\) - радиус скважины, м
\item
  \(S\) - скин-фактор скважины, м
\end{itemize}

    \begin{tcolorbox}[breakable, size=fbox, boxrule=1pt, pad at break*=1mm,colback=cellbackground, colframe=cellborder]
\prompt{In}{incolor}{ }{\boxspacing}
\begin{Verbatim}[commandchars=\\\{\}]

\end{Verbatim}
\end{tcolorbox}


    % Add a bibliography block to the postdoc
    
    
    
\end{document}
